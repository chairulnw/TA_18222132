% ==========================================
% BAB II STUDI LITERATUR
% ==========================================
\chapter{STUDI LITERATUR}
\label{chap:studi-literatur}

\section{Building Management System (BMS)}
Building Management System (BMS) merupakan sistem terpusat yang mengintegrasikan pemantauan dan pengendalian berbagai subsistem gedung seperti HVAC, pencahayaan, energi, dan keamanan dalam satu platform untuk meningkatkan kenyamanan penghuni sekaligus efisiensi energi dan biaya operasional (Hussain dkk. 2025). 

Secara arsitektural, BMS umumnya dibangun dalam tiga lapisan, yaitu low layer yang terdiri atas perangkat seperti sensor, aktuator, dan kontroler. Kemudian ada middle layer yang berisi pengolah data dari perangkat low layer seperti Supervisory Control and Data Acquisition (SCADA) atau Direct Digital Control (DDC). Terakhir adalah top layer yang berisi tampilan penyajian dashboard, analitik, dan pelaporan, yang diolah dari middle layer (Domingues dkk. 2015). 

BMS sering diposisikan sebagai pondasi bangunan cerdas (smart building), karena integrasi teknologi informasi, sensor, dan otomasi yang memungkinkan gedung beroperasi secara efisien berdasarkan data real-time, sekaligus mengelola kinerja energi, kenyamanan, keamanan, dan keselamatan dalam satu kerangka manajemen terintegrasi (Eini dkk. 2022).

Di Indonesia, Bangunan Gedung Cerdas diatur dalam Peraturan Menteri Pekerjaan Umum dan Perumahan Rakyat Nomor 10 Tahun 2023 yang menekankan bahwa bangunan cerdas harus memanfaatkan teknologi otomasi, sensor, dan sistem manajemen terintegrasi untuk mencapai efisiensi energi, keselamatan, kenyamanan, dan keberlanjutan. Regulasi tersebut menempatkan BMS sebagai komponen wajib dalam arsitektur bangunan cerdas karena fungsinya tersebut. Permen ini juga menegaskan bahwa BGC harus memiliki sistem manajemen keamanan dan keselamatan yang terhubung dengan platform pengelolaan gedung, sehingga memastikan kemampuan monitoring, respons, dan pengelolaan risiko yang lebih baik.

\section{Sistem Keamanan Berbasis CCTV}
Sistem kamera pengawas atau closed-circuit television (CCTV) merupakan bagian penting dari sistem keamanan dan keselamatan Bangunan Gedung. Sistem ini berfungsi merekam dan memantau aktivitas pada area-area tertentu melalui kamera yang dipasang secara strategis. Dengan pemantauan yang berlangsung terus-menerus, CCTV memungkinkan petugas keamanan untuk mengidentifikasi aktivitas mencurigakan dan mengambil tindakan secara cepat.

Menurut ketentuan paparan teknis Permen PUPR Nomor 10 Tahun 2023, sistem kamera pengawas terdiri atas tujuh komponen utama, yaitu kamera, stasiun pemantauan, perekam video, jaringan kabel atau router, perangkat penyimpanan rekaman, catu daya utama dan cadangan, serta sistem manajemen video. Kamera yang digunakan dapat berupa kamera IP atau kamera analog, dengan beragam bentuk seperti bullet, dome, dan pan-tilt-zoom (PTZ) yang dipilih sesuai kebutuhan pemantauan di area gedung.

Dalam konteks Bangunan Gedung Cerdas, CCTV bekerja secara terintegrasi dengan sistem keamanan lainnya seperti sistem kontrol akses dan sistem alarm. Integrasi ini menghasilkan mekanisme pengawasan yang lebih komprehensif dan responsif.

\section{Computer Vision}
Computer vision merupakan cabang dari kecerdasan buatan yang berfokus pada bagaimana mesin dapat memahami data visual sehingga mampu mengekstrak informasi untuk pengambilan keputusan (Szeliski 2022). Bidang ini memiliki banyak aplikasinya seperti object detection, object tracking, pose estimation, dan masih banyak lagi.

\subsection{Object Detection}
Object detection merupakan metode untuk mengenali sekaligus menemukan keberadaan suatu objek pada video dan gambar digital (kumar 2024). Menurut Aslam (2020) terdapat dua tipe object detector, yaitu two-stage detector dan one-stage detector.

Pada two-stage detector, seperti Faster R-CNN, proses deteksi dilakukan dalam dua tahap, yaitu menghasilkan kandidat region proposal, kemudian mengklasifikasikan dan menyempurnakan bounding box pada tiap region tersebut. Pendekatan ini biasanya memberikan akurasi tinggi, tetapi memiliki kecepatan inferensi yang lebih lambat karena prosesnya yang bertingkat.

Sedangkan one-stage detector, seperti Single Shot MultiBox Detector (SSD) dan YOLO memproses seluruh gambar secara langsung tanpa tahap terpisah, sehingga klasifikasi dan regresi bounding box dilakukan dalam satu langkah terpadu. pendekatan ini memberikan hasil yang lebih cepat, meskipun akurasinya lebih rendah dibandingkan two-stage detector sehingga pendekatan ini lebih cocok untuk aplikasi real-time.

\subsection{Object Tracking}
Object tracking merupakan proses yang bertujuan untuk menentukan posisi suatu objek secara konsisten di setiap frame dalam sebuah video (praveen 2023). Tujuan utamanya adalah membentuk lintasan (trajectory) pergerakan setiap objek sehingga sistem tidak hanya mengetahui di mana objek muncul pada satu frame, tetapi juga bagaimana objek tersebut bergerak sepanjang waktu.

\subsection{Pose Estimation}
Pose estimation merupakan proses yang bertujuan untuk menemukan lokasi bagian tubuh manusia dan membangun representasi kerangka tubuh (skeleton) dari data masukan seperti gambar atau video. Teknik ini mendeteksi titik-titik kunci (keypoints) seperti sendi bahu, siku, lutut, dan pergelangan tangan, kemudian menghubungkannya menjadi struktur rangka dua dimensi (2D) atau tiga dimensi (3D) (Zheng dkk. 2023).

\section{Aktivitas Mencurigakan}
Aktivitas mencurigakan merujuk pada perilaku yang menyimpang dari pola normal suatu lingkungan dan berpotensi menimbulkan dampak merugikan jika tidak segera ditangani, seperti berlari di area yang seharusnya tenang, berkelahi, berkeliaran terlalu lama, atau membawa objek berbahaya. Dalam literatur, topik ini umumnya dibahas sebagai suspicious human activity recognition, abnormal event detection, atau video anomaly detection dalam konteks sistem pemantauan (Anoopa 2022). Pendekatan deteksi aktivitas mencurigakan telah berkembang dari metode rule-based menjadi pendekatan berbasis deep learning seiring meningkatnya kemampuan komputasi dan ketersediaan dataset video beranotasi.
\section{Metrik Evaluasi}

Evaluasi kinerja model deteksi objek dan aktivitas diperlukan untuk memastikan bahwa sistem mampu menghasilkan prediksi yang akurat dan konsisten dalam lingkungan operasional nyata. Pada sistem pengawasan berbasis CCTV, metrik evaluasi harus mampu mengukur kemampuan model dalam mengidentifikasi objek melalui \textit{bounding box} dan kelas, serta kemampuan dalam mengenali aktivitas atau kejadian tertentu. metrik evaluasi yang sering digunakan meliputi confusion matrix, precision, recall, F1‑score, dan mean Average Precision (mAP) (Zvereva 2025).

\subsection{Confusion Matrix}
Confusion matrix adalah representasi performa model dalam bentuk tabel yang membandingkan hasil prediksi dengan label sebenarnya. Matriks ini terdiri atas empat komponen utama, yaitu True Positive (TP), False Positive (FP), False Negative (FN), dan True Negative (TN). Struktur ini memungkinkan analisis terhadap pola kesalahan model, seperti kecenderungan menghasilkan alarm palsu atau gagal mendeteksi kejadian penting.

\subsection{Precision}
Precision mengukur proporsi prediksi positif yang benar. Nilai precision yang tinggi menunjukkan bahwa model jarang memberikan deteksi positif yang keliru.
\[
\text{Precision} = \frac{TP}{TP + FP}
\]

\subsection{Recall}
Recall mengukur kemampuan model dalam mendeteksi seluruh kejadian positif yang benar.
\[
\text{Recall} = \frac{TP}{TP + FN}
\]
Dalam konteks sistem keamanan, nilai recall yang tinggi sangat penting untuk meminimalkan risiko aktivitas mencurigakan yang tidak terdeteksi.

\subsection{F1-score}
F1-score merupakan rata-rata harmonik dari precision dan recall, memberikan penilaian yang seimbang pada kedua aspek tersebut.
\[
F1 = 2 \cdot \frac{\text{Precision} \cdot \text{Recall}}{\text{Precision} + \text{Recall}}
\]
Metrik ini ideal untuk kondisi dengan distribusi kelas yang tidak seimbang.

\subsection{mean Average Precision (mAP)}
Mean Average Precision (mAP) merupakan metrik evaluasi utama pada deteksi objek. Metrik ini menghitung rata-rata presisi pada berbagai nilai ambang Intersection over Union (IoU). mAP menilai kemampuan model dalam memprediksi kelas objek sekaligus menghasilkan \textit{bounding box} yang akurat. Penggunaan mAP relevan dalam penelitian ini karena performa deteksi objek menjadi komponen dasar dalam analisis perilaku dan aktivitas mencurigakan.


\section{Penelitian Terkait}
\lipsum[19-20]
